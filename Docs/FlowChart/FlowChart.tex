
\section{Overview of a single step}
\label{flow:sec:nosdc}

The main evolution for a single step is contained in
\code{Castro\_advance.cpp}, as \code{Castro::advance()}.  This does
the following advancement.  Note, some parts of this are only done
depending on which preprocessor directives are defined at
compile-time---the relevant directive is noted in the [\ ] at the start
of each step.

\begin{enumerate}
\item {\em Initialization} 

  This sets up the current level for advancement.  If we are at the
  start of a coarse level timestep, or we are in the middle of
  subcycling on a finer level (\code{amr\_iteration} {\tt > 1}), then
  we swap the \statedata\ from the new to old (e.g., this ensures that
  the next evolution starts with the result from the previous step).
  Any local storage arrays are allocated here.

\item {\em Advancement} 

  This calls \code{do\_advance} to take a single hydrodynamics step.
  The advance is done with a method of lines integration.

\item {\em Finalize}

  This cleans up the memory used during the step.  

\end{enumerate}

The system of equations is:
\begin{equation}
\frac{\partial\Ub}{\partial t} = \nabla\cdot\Fb + \Sb,
\end{equation}
where $\Fb$ is the flux vector, and $\Sb$ are the non-reaction source terms.
The update is given by:
\begin{align}
  \Ub^{n+1} &= \Ub^n - \dt\, {\bf A}(\Ub^\star) + \dt\, \Sb(\Ub^\star)
\end{align}
The construction of the advective terms, ${\bf A(\Ub)}$ is purely
explicit, and based on a second-order Godunov method.  We
predict the standard primitive variables, as well as $\rho e$, at
zone edges and use an approximate Riemann solver construct
fluxes.

Here is the single-level algorithm.  The goal here is to update the
\variable{State\_Type} \statedata\ from the old to new time (see
\S~\ref{soft:sec:statedata}).  We will use the following notation
here, consistent with the names used in the code:
\begin{itemize}
\item \variable{S\_old} is a multifab reference to the old-time-level
  {\tt State\_Type} data.

\item \variable{Sborder} is a multifab that has ghost cells and is
  initialized from {\tt S\_old}.  This is what the hydrodynamic
  reconstruction will work from.

\item \variable{S\_new} is a multifab reference to the new-time-level
  {\tt State\_Type} data.
\end{itemize}

In the code, the objective is to evolve the state from the old time,
{\tt S\_old}, to the new time, {\tt S\_new}.

\begin{enumerate}
\item \label{strang:init} {\em Initialize}

  \begin{enumerate}
  \item In \code{initialize\_do\_advance()} :
    \begin{enumerate}
    \item Reset the flux registers and initializes a lot of
      intermediate storage arrays (like source term) to zero.

    \item Create \variable{Sborder}, initialized from {\tt S\_old}
    \end{enumerate}

  \item Check for NaNs in the initial state, {\tt S\_old}.
  \end{enumerate}

\item \label{strang:hydro} {\em Construct the hydro update} [\code{construct\_hydro\_source()}]

  The goal is to advance our system considering only the advective
  terms (which in Cartesian coordinates can be written as the
  divergence of a flux).

  We do the hydro update in two parts---first we construct the
  advective update and store it in the \variable{hydro\_source}
  \multifab, then we do the conservative update in a separate step.  This
  separation allows us to use the advective update separately in more
  complex time-integration schemes.

  The advection step is complicated, and more
  detail is given in Section \ref{Sec:Advection Step}.  Here is the
  summarized version:
  \begin{enumerate}
  \item Compute primitive variables.
  \item Predict primitive variables to zone edges.
  \item Solve the Riemann problem.
  \item Compute fluxes and update.
  \end{enumerate}

  The update computed here is then immediately applied to
  \variable{S\_new}.

\item \label{strang:clean} {\em Clean State} [\code{clean\_state()}]

  There are many ways that the hydrodynamics state may become
  unphysical in the evolution.  The \code{clean\_state()} routine
  enforces some checks on the state.  In particular, it
  \begin{enumerate}
  \item enforces that the density is above \runparam{castro.small\_dens}
  \item normalizes the species so that the mass fractions sum to 1
  \item resets the internal energy if necessary (too small or negative)
    and computes the temperature for all zones to be thermodynamically 
    consistent with the state.
  \end{enumerate}
  This is done on \variable{S\_new}.

  After these checks, we check the state for NaNs.
 
\end{enumerate}




