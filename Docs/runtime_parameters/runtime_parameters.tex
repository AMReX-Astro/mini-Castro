\section{ {\tt castro } Namespace}

\label{ch:parameters}


%%%%%%%%%%%%%%%%
% symbol table
%%%%%%%%%%%%%%%%

\begin{landscape}


{\small

\renewcommand{\arraystretch}{1.5}
%
\begin{center}
\begin{longtable}{|l|p{5.25in}|l|}
\caption[castro parameters]{castro parameters} \label{table: castro parameters runtime} \\
%
\hline \multicolumn{1}{|c|}{\textbf{parameter}} & 
       \multicolumn{1}{ c|}{\textbf{description}} & 
       \multicolumn{1}{ c|}{\textbf{default value}} \\ \hline 
\endfirsthead

\multicolumn{3}{c}%
{{\tablename\ \thetable{}---continued}} \\
\hline \multicolumn{1}{|c|}{\textbf{parameter}} & 
       \multicolumn{1}{ c|}{\textbf{description}} & 
       \multicolumn{1}{ c|}{\textbf{default value}} \\ \hline 
\endhead

\multicolumn{3}{|r|}{{\em continued on next page}} \\ \hline
\endfoot

\hline 
\endlastfoot


\rowcolor{tableShade}
\runparamNS{cfl}{castro} &  the effective Courant number to use---we will not allow the hydrodynamic waves to cross more than this fraction of a zone over a single timestep & 0.8 \\
\runparamNS{small\_dens}{castro} &  the small density cutoff.  Densities below this value will be reset & -1.e200 \\
\rowcolor{tableShade}
\runparamNS{small\_temp}{castro} &  the small temperature cutoff.  Temperatures below this value will be reset & -1.e200 \\


\end{longtable}
\end{center}

} % ends \small


\end{landscape}

%


