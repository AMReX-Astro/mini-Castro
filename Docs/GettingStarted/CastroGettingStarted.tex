
\section{Downloading the Code}

\castro\ is built on top of the \amrex\ framework.  In order to run
\castro\, you must download two separate git modules.

\vspace{.1in}

\noindent First, make sure that {\tt git} is installed on your machine---we recommend version 1.7.x or higher.

\vspace{.1in}

\begin{enumerate}

\item Clone/fork the \amrex\ repository from the {\tt AMReX-Codes} {\sf
  github} page (\url{https://github.com/AMReX-Codes/amrex/}).  To
  clone via the command line, simply type:
\begin{verbatim}
git clone https://github.com/AMReX-Codes/amrex.git
\end{verbatim}
Alternately, if you have a {\sf github} account with your
machine's SSH-keys registered, you can do:
\begin{verbatim}
git clone ssh://git@github.com/AMReX-Codes/amrex.git
\end{verbatim}

This will create a directory called {\tt amrex/} on your machine.

You will want to periodically update \amrex\ by typing
\begin{verbatim}
git pull
\end{verbatim}
in the {\tt amrex/} directory.  

Note: actively development is done on the {\tt development} branch
in each repo, and merged into the {\tt master} branch periodically.
If you wish to use the \castro\ {\tt development} branch, then you
should also switch to the {\tt development} branch for \amrex.

\item Set the environment variable, {\tt AMREX\_HOME}, on your
  machine to point to the path name where you have put \amrex.
  You can add this to your {\tt .bashrc} as:
\begin{Verbatim}[commandchars=\\\{\}]
export AMREX_HOME={\em /path/to/amrex/}
\end{Verbatim}
where you replace \texttt{\em /path/to/amrex/} will the full path to the
{\tt amrex/} directory.

\item Clone/fork the \castro\ repository from the same {\sf
  github} organization as above, using either HTTP access:
\begin{verbatim}
git clone https://github.com/AMReX-Astro/Castro.git
\end{verbatim}
or SSH access if you have it enabled:
\begin{verbatim}
git clone ssh://git@github.com:/AMReX-Astro/Castro.git
\end{verbatim}
Or, as above, you can download a ZIP file of the code from
\href{https://github.com/AMReX-Astro}{our main {\sf github} page},
by clicking on the \castro\ link.

As with \amrex, development on \castro\ is done in the
{\tt development} branch, so you should work there if you want
the latest source.

\item We recommend setting the {\tt CASTRO\_HOME} environment
  variable to point to the path name where you have put \castro.
  Add the following to your {\tt .bashrc}:
\begin{verbatim}
export CASTRO_HOME="/path/to/Castro/"
\end{verbatim}

\end{enumerate}

%\clearpage

\section{Building the Code}

You can build and run the problem in the {\tt Exec/} subdirectory.
Simply type {\tt make}. The resulting executable will look something like {\tt
  Castro3d.Linux.gnu.MPI.OMP.ex}, which means this is 3-d code,
  made on a Linux machine, compiled with the GNU compilers,
  using OpenMP and MPI for parallelization.

  In {\tt Exec/}, you can edit the {\tt GNUmakefile}, to change
  a few options, or set these at compile time as arguments to {\tt make}:
  \begin{itemize}

    \item {\tt COMP = gnu}

      This is the set of compilers.  {\tt gnu} are a good default
      choice (this will use {\tt g++} and {\tt gfortran}.  You can
      also choose {\tt pgi} and {\tt intel} for example.

      If you want to try other compilers than the GNU suite and they
      don't work, please let us know.

    \item {\tt DEBUG = FALSE}

      This disables debugging checks and results in a more
      optimized executable.

    \item {\tt USE\_MPI = TRUE}

      This turns on parallelization via MPI. This requires that you have the MPI library
      installed on your machine.  In this case, the build system will
      need to know about your MPI installation.  This can be done by
      editing the makefiles in the \amrex\ tree, but the default
      fallback is to look for the standard MPI wrappers (e.g.\ {\tt
        mpic++} and {\tt mpif90}) to do the build.

    \item {\tt USE\_OMP = TRUE}

      This turns on parallelization with OpenMP.

  \end{itemize}

\section{Running the Code}

\begin{enumerate}

\item \castro\ takes an input file that overrides the runtime parameter defaults.
  The code is run as:
\begin{verbatim}
Castro3d.Linux.gnu.MPI.OMP.ex inputs.3d.sph
\end{verbatim}

This will run the spherical Sedov problem in Cartesian ($x$-$y$-$z$
coordinates).  You can see other possible options, which should be
clear by the names of the inputs files.

\item You will notice that running the code generates directories that
  look like {\tt plt00000/}, {\tt plt00020/}, etc. These are ``plotfiles'',
  which are used for visualization and analysis.

\end{enumerate}

\section{Visualization of the Results}
\index{visualization}

There are several options for visualizing the data.  The popular
\visit\ package supports the \amrex\ file format natively, as does the
\yt\ python package\footnote{Each of these will recognize it as the
  \boxlib\ format.}.  The standard tool used within the
\boxlib-community is \amrvis, which we demonstrate here.

\begin{enumerate}

\item Get \amrvis:

\begin{verbatim}
git clone https://ccse.lbl.gov/pub/Downloads/Amrvis.git
\end{verbatim}

Then cd into {\tt Amrvis/}, edit the {\tt GNUmakefile} there
to set {\tt DIM = 3}, and again set {\tt COMP} to compilers that
you have. Leave {\tt DEBUG = FALSE}.

Type {\tt make} to build, resulting in an executable that
looks like {\tt amrvis2d...ex}.

You will also need to obtain and build {\tt volpack}:
\begin{verbatim}
git clone https://ccse.lbl.gov/pub/Downloads/volpack.git
\end{verbatim}

Note: \amrvis\ requires the OSF/Motif libraries and headers. If you don't have these 
you will need to install the development version of motif through your package manager. 
{\tt lesstif} gives some functionality and will allow you to build the amrvis executable, 
but \amrvis\ may not run properly.

On most Linux distributions, the motif library is provided by the
{\tt openmotif} package, and its header files (like {\tt Xm.h}) are provided
by {\tt openmotif-devel}. If those packages are not installed, then use the
package management tool to install them, which varies from
distribution to distribution, but is straightforward. 

You may want to create an alias to {\tt amrvis3d}, for example
\begin{verbatim}
alias amrvis3d /tmp/Amrvis/amrvis3d...ex
\end{verbatim}
where {\tt /tmp/Amrvis/amrvis3d...ex} is the full path and name of the \amrvis\ executable.

\item Configure \amrvis:  

  Copy the {\tt amrvis.defaults} file to your home directory (you can
  rename it to {\tt .amrvis.defaults} if you wish).  Then edit the
  file, and change the {\tt palette} line to point to the full
  path/filename of the {\tt Palette} file that comes with \amrvis.

\item Visualize:

  Return to the {\tt Castro/Exec/hydro\_tests/Sedov} directory.  You should
  have a number of output files, including some in the form {\tt *pltXXXXX},
  where {\tt XXXXX} is a number corresponding to the timestep the file
  was output.  {\tt
    amrvis2d {\em filename}} to see a single plotfile, or {\tt amrvis2d -a
  *plt*}, which will animate the sequence of plotfiles.

  Try playing
  around with this---you can change which variable you are
  looking at, select a region and click ``Dataset'' (under View)
  in order to look at the actual numbers, etc. You can also export the
  pictures in several different formats under "File/Export".

Please know that we do have a number of conversion routines to other
formats (such as matlab), but it is hard to describe them all. If you
would like to display the data in another format, please let us know
and we will point you to whatever we have that can help.

\end{enumerate}
